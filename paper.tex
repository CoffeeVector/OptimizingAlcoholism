\documentclass[letterpaper, 12pt]{article}
\usepackage[margin=1in]{geometry}
\usepackage{amsmath}
\usepackage{pgfplots}
\usepackage{listings}
\pgfplotsset{width=15cm,compat=1.9} \newcommand{\todo}[1]{{\emph{\color{red}#1}}}
\title{The Optimization of Alcoholism under a Hypothetical Bartering System}
\author{Kevin Palani and Kevin Zheng}
\begin{document}
\maketitle
\tableofcontents
\section{Introduction to the hypothetical bartering system}
\par You have \$10, and a beer is 2\$.
Very quickly you can see that if you spend all 10\$, you will get 5 beers.
Once you've drunken the 5 beers, you are left with 5 beer bottles and 5 caps.
The store owner strikes you a deal.
If you give him two bottles, he'll give you a new fresh bottle of beer.
You can give him four bottle caps and he'll also give you a new fresh bottle of beer.
He is also kind enough to let your drink before you pay.
You are interested in how many drinks you can get.
\section{Modeling amount of caps and bottles}
\subsection{Vector representation of caps and bottles}
\[
	\begin{bmatrix}
		a\\
		b\\
		1\\
	\end{bmatrix}
\]
\par Will be the vector that represents the bottles and caps such that $a$ is the amount of bottles, and $b$ is the amount of caps.
$1$ is the homogenization of the vector.
\subsection{Matrix representation of the bartering system}
\par If we can spend two empty bottles and receive a full drink, that is equivalent to spending two bottles and getting one bottle and one cap.
We will represent this operation as the following translational matrix.
\[
	B=
	\begin{bmatrix}
		1 && 0 && -2 + 1\\
		0 && 1 &&  1\\
		0 && 0 &&  1\\
	\end{bmatrix}
\]
\par And since we can drink before we pay, having merely one empty bottle is enough to drink.
\[
	B=
	\begin{bmatrix}
		1 && 0 && -1\\
		0 && 1 &&  1\\
		0 && 0 &&  1\\
	\end{bmatrix}
\]
\par The purchasing of a full drink using 4 caps can be similarly represented as a translational matrix.
\[
	C=
	\begin{bmatrix}
		1 && 0 &&  1\\
		0 && 1 && -4 + 1\\
		0 && 0 &&  1\\
	\end{bmatrix}
\]
\par Which can simply be evaluated to.
\[
	C=
	\begin{bmatrix}
		1 && 0 &&  1\\
		0 && 1 && -3\\
		0 && 0 &&  1\\
	\end{bmatrix}
\]
\par These two operations can be represented geometrically as a translation of a point on a 2 dimensional Cartesian plane.
For example, if we start with 5 empty bottles and 5 empty caps, we can trace the motion of the point as following.
\begin{center}
	\begin{tikzpicture}
		\begin{axis}[
				axis lines = left,
			xlabel = Bottles,
			ylabel = {Caps},
			xmin = 0, xmax = 10,
			ymin = 0, ymax = 10
			]
			\addplot [
				domain=0:5,
			samples=100,
			color=blue,
			]
			{-1 * (x - 5) + 5};
			\addlegendentry{Trading empty bottles for drinks}
			\addplot [
				domain=0:3,
			samples=100,
			color=red,
			]
			{-3 * x + 10};
			\addlegendentry{Trading caps for drinks}
			\addplot [
				domain=0:3,
			samples=100,
			color=blue,
			]
			{-1 * (x - 3) + 1};
			\addplot [
				domain=0:1,
			samples=100,
			color=red,
			]
			{-3 * x + 4};
			\addplot [
				domain=0:1,
			samples=100,
			color=blue,
			]
			{-1 * (x - 1) + 1};
			\addplot[
				color=blue,
			only marks,
			mark=square,
					]
			coordinates {
				(0,2)(5,5)
			};
		\end{axis}
	\end{tikzpicture}
\end{center}
\section{Proof of the final state of the vector}
\subsection{Algorithm}
\lstinputlisting[language=python]{finalstate.py}
\subsection{Linear algebra}
The final state can be represented as the following:
\begin{align*}
\begin{bmatrix}
	5\\
	5
\end{bmatrix}
+ c
\begin{bmatrix}
	-3\\
	1	
\end{bmatrix}
+ b
\begin{bmatrix}
	1\\
	-1	
\end{bmatrix}
 = 
\begin{bmatrix}
	2\\
	0
\end{bmatrix}
\end{align*}
Which seems to be some kind of reformulation of modular arithmetic for vectors.
This can also be thought of as:
\begin{align*}
c
\begin{bmatrix}
	-3\\
	1	
\end{bmatrix}
+ b
\begin{bmatrix}
	1\\
	-1	
\end{bmatrix}
=
\begin{bmatrix}
	2\\
	0
\end{bmatrix}
 - 
\begin{bmatrix}
	5\\
	5
\end{bmatrix}
\end{align*}
Which simplifies to:
\begin{align*}
\begin{bmatrix}
	-3 & 1\\
	-1 & 1	
\end{bmatrix}
\begin{bmatrix}
	c\\
	b
\end{bmatrix}
=
\begin{bmatrix}
	2\\
	0
\end{bmatrix}
 - 
\begin{bmatrix}
	5\\
	5
\end{bmatrix}
\\
\begin{bmatrix}
	c\\
	b
\end{bmatrix}
=
\begin{bmatrix}
	-\frac{1}{2} & -\frac{1}{2}\\
	-\frac{1}{2} & -\frac{3}{2}
\end{bmatrix}
(
\begin{bmatrix}
	2\\
	0
\end{bmatrix}
 - 
\begin{bmatrix}
	5\\
	5
\end{bmatrix}
)
\end{align*}
\par Such that $b$ and $c$ are integer solutions.
Which is the largest vector that can be represented as an integer linear combination of the basis vectors that represent the trades.
\subsection{WIP: Ideas}
\begin{itemize}
	\item GCD? The algorithm is quite similar to Euclid's algorithm for greatest common denominator
\end{itemize}
\section{Using the final state of the vector to deduce the amount of drinks one had}
\par Since two empty bottles can get you a drink and a drink is worth \$2, then that means one bottle is worth \$1.
Similarly since four bottle caps can get you a drink and a drink is worth \$2, then that means bottle cap is worth \$0.5.
\par Since a full drink is consisted of one cap, one bottle, and some drink, we can use simple algebra to deduce that:
\begin{equation}
	\$2 = d + \$1 + \$0.5
\end{equation}
\begin{equation}
	d = \$0.5
\end{equation}
the worth of the drink is \$0.5.
If we started with \$10 dollars, and we are left with $a$ bottles and $b$ caps, then:
\begin{equation}
	\$10 = \$0.5x + \$a + \$0.5b
\end{equation}
\begin{equation}
	x = \frac{\$10 - \$a - \$0.5b}{\$0.5}
\end{equation}
\end{document}
