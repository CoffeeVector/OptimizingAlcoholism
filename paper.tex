\documentclass[letterpaper, 12pt]{article}
\usepackage[margin=1in]{geometry}
\usepackage{amsmath}
\usepackage{pgfplots}
\usepackage{listings}
\usepackage{graphicx}
\usepackage{amsmath}
\usepackage{amssymb}
\usepackage{amsthm}
\pgfplotsset{width=15cm,compat=1.9} \newcommand{\todo}[1]{{\emph{\color{red}#1}}}
\title{WIP: The Optimization of Alcoholism under a Hypothetical Bartering System}
\author{Kevin Palani and Kevin Zheng}
\begin{document}
\maketitle
\tableofcontents
\section{Introduction to the hypothetical bartering system}
\par You have \$10, and a beer is 2\$.
Very quickly you can see that if you spend all 10\$, you will get 5 beers.
Once you've drunken the 5 beers, you are left with 5 beer bottles and 5 caps.
The store owner strikes you a deal.
If you give him two empty bottles or four bottle caps, he'll give you a new bottle of beer.
He is also kind enough to let you drink before you pay.
How many drinks you can get?
\section{Naive solution}
A naive solution would be to consider the worth of the beer itself.
Since 2 bottles can get a \$2 beer, that means a bottle is worth \$1.
Since 4 caps can get a \$2 beer, that means a cap is worth \$0.50.
A beer is consisted of a bottle, a cap, and some drink.
Therefore the drink itself is worth \$0.50.
If we start with \$10, we can get 20 drinks.

The problem with this solution is that this assumes that in the end we can successfully convert all of our bottles and caps into booze, but can we?
\section{Modeling amount of caps and bottles}
\subsection{Vector representation of caps and bottles}
\[
    \begin{bmatrix}
        a\\
        b\\
    \end{bmatrix}
\]
\par Will be the vector that represents the bottles and caps such that $a$ is the amount of bottles, and $b$ is the amount of caps.
\subsection{Representation of the bartering system}
\par If we can spend two empty bottles and receive a full drink, that is equivalent to spending two bottles and getting one bottle and one cap.
We will represent this operation as the addition of the following vector;
\[
    \begin{bmatrix}
       -2 + 1\\
       1\\
    \end{bmatrix}
\]
\par And since we can drink before we pay, having only one empty bottle is enough to drink.
\[
    \begin{bmatrix}
       -1\\
        1\\
    \end{bmatrix}
\]
\par Using 4 caps can be represented similarly
\[
    \begin{bmatrix}
        1\\
        -4 + 1\\
    \end{bmatrix}
\]
\par Which can simply be evaluated to.
\[
    \begin{bmatrix}
         1\\
        -3\\
    \end{bmatrix}
\]
\par These two operations can be represented geometrically as a translation of a point on a 2 dimensional Cartesian plane.
For example, if we start with 5 empty bottles and 5 empty caps, we can trace the motion of the point as following.
\begin{center}
    \begin{tikzpicture}
        \begin{axis}[
                axis lines = left,
            xlabel = Bottles,
            ylabel = {Caps},
            xmin = 0, xmax = 10,
            ymin = 0, ymax = 10
            ]
            \addplot [
                domain=0:5,
            samples=100,
            color=blue,
            ]
            {-1 * (x - 5) + 5};
            \addlegendentry{Trading empty bottles for drinks}
            \addplot [
                domain=0:3,
            samples=100,
            color=red,
            ]
            {-3 * x + 10};
            \addlegendentry{Trading caps for drinks}
            \addplot [
                domain=0:3,
            samples=100,
            color=blue,
            ]
            {-1 * (x - 3) + 1};
            \addplot [
                domain=0:1,
            samples=100,
            color=red,
            ]
            {-3 * x + 4};
            \addplot [
                domain=0:1,
            samples=100,
            color=blue,
            ]
            {-1 * (x - 1) + 1};
            \addplot[
                color=blue,
            only marks,
            mark=square,
                    ]
            coordinates {
                (0,2)(5,5)
            };
        \end{axis}
    \end{tikzpicture}
\end{center}
\section{Analysis of the graph}
\par We know from the context of the problem, that the final state can only be the following points:
$$(0, 0)$$
$$(0, 1)$$
$$(0, 2)$$
Because having either 1 bottle, or 3 caps is enough to do another transaction.
\par Visually, you can already tell that it's impossible to get to the point $(0, 0)$ unless if you have already started there (sorry mate, you gotta buy some beer to play the game).
\par You can also tell that you cannot get to $(0, 1)$, because that means you came from $(1, 0)$ through a blue line, but that means you came from $(0, 2)$ from a red line, which is impossible because $(0, 2)$ is not enough to continue a transaction.
\par So already from this graph, you can tell that you will always end up with 2 caps left over (if we ignore the trivial case that you do not buy beer in the first place).
%\section{Computing the final state of bartering}
%\subsection{Algorithm}
%\lstinputlisting[language=python]{./python/finalstate1.py}
%\begin{center}
%    \includegraphics[width=0.49\textwidth]{./python/bots1.png}
%    \includegraphics[width=0.49\textwidth]{./python/caps1.png}
%\end{center}
%From this it makes sense that you'd never end up with a bottle since having one left would always allow you to get another one (boohoo, special case).
%The repeating structure of the bottle caps, while not what you may initially think, is not too hard to figure out.
\subsection{Algebraic Solution}
Let $\vec{i}$ be the initial state, $\vec{f}$ be the final state, $c$ be the cap-based transactions, and $b$ be the bottle-based transaction.
\begin{align*}
    \vec{i}
    + c
    \begin{bmatrix}
        -3\\
        1
    \end{bmatrix}
    + b
    \begin{bmatrix}
        1\\
        -1
    \end{bmatrix}
    &=
    \vec{f}\\
    c
    \begin{bmatrix}
        -3\\
        1
    \end{bmatrix}
    + b
    \begin{bmatrix}
        1\\
        -1
    \end{bmatrix}
    &=
    \vec{f} - \vec{i}\\
    \begin{bmatrix}
        -3 & 1\\
         1 &-1
    \end{bmatrix}
    \begin{bmatrix}
        c\\
        b
    \end{bmatrix}
    &=
    \vec{f} - \vec{i}\\
    \begin{bmatrix}
        c\\
        b
    \end{bmatrix}
    &=
    \frac{1}{2}
    \begin{bmatrix}
        -1 &-1\\
        -1 &-3
    \end{bmatrix}
    (\vec{f} - \vec{i})\\
\end{align*}
The goal is to get as drunk as possible, which is reaching as many transactions as possible, thus we want to maximize $b + c$ with respect to the elements of $f$.
The above matrix equation can be represented as the following set of equations.
\begin{align*}
    c &= \frac{1}{2}(-(f_c - i_c) - (f_b - i_b))\\
    b &= \frac{1}{2}(-(f_c - i_c) - 3(f_b - i_b))\\
    c + b &= \frac{1}{2}(-2(f_c - i_c) - 4(f_b - i_b))
\end{align*}
Which can be simplified to:
\begin{align*}
    c &= \frac{1}{2}(-f_c - f_b  + i_c + i_b))\\
    b &= \frac{1}{2}(-f_c - 3f_b + i_c + i_b))\\
    c + b &= (-f_c - 2f_b + i_c + 2i_b)
\end{align*}
Using the typical optimization with derivatives isn't going to help us here, because the function is linear with respect to both $f_c$ and $f_b$.
Instead we can use the constraint that $c$, $b$, and $c + b$ must be natural numbers (in my definition, the natural numbers include 0).
\begin{align*}
    c &\in \mathbb{N} \\
    -f_c - f_b + i_b + i_c &\in 2\mathbb{N}
\end{align*}
Since $i_b = i_c$, $i_b + i_c \in 2\mathbb{N}$
\begin{align*}
    -f_c - f_b &\in 2\mathbb{Z}\\
    f_c + f_b &\in 2\mathbb{N}
\end{align*}
Thus, we can say that $f_c$ and $f_b$ have the same parity.
\section{Using the final state of the vector to deduce the amount of drinks drunk}
\par If we refer back to the naive solution, we concluded that beer itself is \$0.50.
If we know what we have left over, we can deduce how much of our money was effectively turned into booze.
If we started with \$10 dollars, and we are left with $a$ bottles and $b$ caps, then:
$$10 = 0.5x + a + 0.5b$$
$$x = \frac{10 - a - 0.5b}{0.5}$$
\begin{center}
    or 
\end{center}
\begin{equation}
x = 20 - 2a - b
\end{equation}
\end{document}
